% generated by GAPDoc2LaTeX from XML source (Frank Luebeck)
\documentclass[a4paper,11pt]{report}

            \usepackage{a4wide}
            \newcommand{\bbZ}{\mathbb{Z}}
        
\usepackage[top=37mm,bottom=37mm,left=27mm,right=27mm]{geometry}
\sloppy
\pagestyle{myheadings}
\usepackage{amssymb}
\usepackage[utf8]{inputenc}
\usepackage{makeidx}
\makeindex
\usepackage{color}
\definecolor{FireBrick}{rgb}{0.5812,0.0074,0.0083}
\definecolor{RoyalBlue}{rgb}{0.0236,0.0894,0.6179}
\definecolor{RoyalGreen}{rgb}{0.0236,0.6179,0.0894}
\definecolor{RoyalRed}{rgb}{0.6179,0.0236,0.0894}
\definecolor{LightBlue}{rgb}{0.8544,0.9511,1.0000}
\definecolor{Black}{rgb}{0.0,0.0,0.0}

\definecolor{linkColor}{rgb}{0.0,0.0,0.554}
\definecolor{citeColor}{rgb}{0.0,0.0,0.554}
\definecolor{fileColor}{rgb}{0.0,0.0,0.554}
\definecolor{urlColor}{rgb}{0.0,0.0,0.554}
\definecolor{promptColor}{rgb}{0.0,0.0,0.589}
\definecolor{brkpromptColor}{rgb}{0.589,0.0,0.0}
\definecolor{gapinputColor}{rgb}{0.589,0.0,0.0}
\definecolor{gapoutputColor}{rgb}{0.0,0.0,0.0}

%%  for a long time these were red and blue by default,
%%  now black, but keep variables to overwrite
\definecolor{FuncColor}{rgb}{0.0,0.0,0.0}
%% strange name because of pdflatex bug:
\definecolor{Chapter }{rgb}{0.0,0.0,0.0}
\definecolor{DarkOlive}{rgb}{0.1047,0.2412,0.0064}


\usepackage{fancyvrb}

\usepackage{mathptmx,helvet}
\usepackage[T1]{fontenc}
\usepackage{textcomp}


\usepackage[
            pdftex=true,
            bookmarks=true,        
            a4paper=true,
            pdftitle={Written with GAPDoc},
            pdfcreator={LaTeX with hyperref package / GAPDoc},
            colorlinks=true,
            backref=page,
            breaklinks=true,
            linkcolor=linkColor,
            citecolor=citeColor,
            filecolor=fileColor,
            urlcolor=urlColor,
            pdfpagemode={UseNone}, 
           ]{hyperref}

\newcommand{\maintitlesize}{\fontsize{50}{55}\selectfont}

% write page numbers to a .pnr log file for online help
\newwrite\pagenrlog
\immediate\openout\pagenrlog =\jobname.pnr
\immediate\write\pagenrlog{PAGENRS := [}
\newcommand{\logpage}[1]{\protect\write\pagenrlog{#1, \thepage,}}
%% were never documented, give conflicts with some additional packages

\newcommand{\GAP}{\textsf{GAP}}

%% nicer description environments, allows long labels
\usepackage{enumitem}
\setdescription{style=nextline}

%% depth of toc
\setcounter{tocdepth}{1}





%% command for ColorPrompt style examples
\newcommand{\gapprompt}[1]{\color{promptColor}{\bfseries #1}}
\newcommand{\gapbrkprompt}[1]{\color{brkpromptColor}{\bfseries #1}}
\newcommand{\gapinput}[1]{\color{gapinputColor}{#1}}


\begin{document}

\logpage{[ 0, 0, 0 ]}
\begin{titlepage}
\mbox{}\vfill

\begin{center}{\maintitlesize \textbf{ GRPS1024 \mbox{}}}\\
\vfill

\hypersetup{pdftitle= GRPS1024 }
\markright{\scriptsize \mbox{}\hfill  GRPS1024  \hfill\mbox{}}
{\Huge \textbf{ Library of the groups of order 1024 of p-class at least 3, those of p-class
two and rank 4 and those of p-class 1 and rank 10. \mbox{}}}\\
\vfill

{\Huge  0.0.1 \mbox{}}\\[1cm]
{ 20 June 2022 \mbox{}}\\[1cm]
\mbox{}\\[2cm]
{\Large \textbf{ David Burrell\\
   \mbox{}}}\\
\hypersetup{pdfauthor= David Burrell\\
   }
\end{center}\vfill

\mbox{}\\
{\mbox{}\\
\small \noindent \textbf{ David Burrell\\
   }  Email: \href{mailto://davidburrell@ufl.edu} {\texttt{davidburrell@ufl.edu}}\\
  Homepage: \href{https://davidburrell.github.io/} {\texttt{https://davidburrell.github.io/}}}\\
\end{titlepage}

\newpage\setcounter{page}{2}
\newpage

\def\contentsname{Contents\logpage{[ 0, 0, 1 ]}}

\tableofcontents
\newpage

     
\chapter{\textcolor{Chapter }{Groups of Order 1024}}\label{Chapter_Groups_of_Order_1024}
\logpage{[ 1, 0, 0 ]}
\hyperdef{L}{X850EDC61865B454E}{}
{
  
\section{\textcolor{Chapter }{Overview}}\label{Chapter_Groups_of_Order_1024_Section_Overview}
\logpage{[ 1, 1, 0 ]}
\hyperdef{L}{X8389AD927B74BA4A}{}
{
  

 This library gives explicit access to the following groups of order 1024: 
\begin{itemize}
\item  The Rank 1 group 
\item  All Rank 2 groups 
\item  All Rank 3 groups 
\item  All Rank 4 groups 
\item  Rank 5 groups with p-class at least 3 
\item  Rank 6 groups with p-class at least 3 
\item  Rank 7 groups with p-class at least 3 
\item  Rank 8 groups with p-class at least 3 
\item  Rank 9 groups with p-class at least 3 
\item  The Rank 10 group 
\end{itemize}
 

 This library gives partial information on the remaining groups of order 1024: 
\begin{itemize}
\item  Rank 5 groups with p-class 2 
\item  Rank 6 groups with p-class 2 
\item  Rank 7 groups with p-class 2 
\item  Rank 8 groups with p-class 2 
\item  Rank 9 groups with p-class 2 
\end{itemize}
 

 For the groups that are not explicitly available the following information is
available: 
\begin{itemize}
\item  Parent Group ID 
\item  Parent Group Order 
\item  p-class 
\item  Rank 
\item  Age 
\end{itemize}
 

 The groups are sorted first by their parent group ids and then by the pc codes
of the standard presentations for the groups. The data contained in this
library was used in the 2021 enumeration of the groups of order 1024 \cite{Burrell2021a}. The available groups were generated using the p-group generation algorithm \cite{OBrien1990a} as implemented in the ANUPQ package \cite{Gamble2019a}. The information on the remaining groups was calculated using the
cohomological methods for enumerating p-groups of p-class 2 as introduced in \cite{Eick1999a}. 

 }

 }

   
\chapter{\textcolor{Chapter }{Functionality}}\label{Chapter_Functionality}
\logpage{[ 2, 0, 0 ]}
\hyperdef{L}{X87F1120883F5B4D0}{}
{
  

 
\section{\textcolor{Chapter }{Methods}}\label{Chapter_Functionality_Section_Methods}
\logpage{[ 2, 1, 0 ]}
\hyperdef{L}{X8606FDCE878850EF}{}
{
  

 Once the package is loaded the user may call \texttt{SmallGroup(1024,i)} and receive either a group if available or a partially constructed group which
has the following attributes set 
\begin{itemize}
\item  p-class 
\item  Rank 
\item  Heritage 
\item  Order 
\end{itemize}
 

\subsection{\textcolor{Chapter }{AvailableMap}}
\logpage{[ 2, 1, 1 ]}\nobreak
\hyperdef{L}{X7E989FA979E72135}{}
{\noindent\textcolor{FuncColor}{$\triangleright$\enspace\texttt{AvailableMap({\mdseries\slshape N})\index{AvailableMap@\texttt{AvailableMap}}
\label{AvailableMap}
}\hfill{\scriptsize (function)}}\\
\textbf{\indent Returns:\ }
\texttt{int} 



 Since some of the groups of order 1024 are not available, this function
handles the translation between the ordering of the groups of order 1024 and
the available groups of order 1024. For $1 \leq i \leq 683,875,133$ this will return the position of the $i$th available group among all the groups of order 1024. }

 
\begin{Verbatim}[commandchars=!@|,fontsize=\small,frame=single,label=Example]
  #groups 1-3567 are available SmallGroup(1024,3568) is not available
  !gapprompt@gap>| !gapinput@g:=SmallGroup(1024,3567);|
  <pc group of size 1024 with 10 generators> #this is an available group
  !gapprompt@gap>| !gapinput@g:=SmallGroup(1024,3568);|
  <pc group with 0 generators> #this is a partially constructed group and not available
  #the next available group has index 378632399 
  !gapprompt@gap>| !gapinput@AvailableMap(3568)|
  378632399
  #To iterate through the available groups use
  !gapprompt@gap>| !gapinput@SmallGroup(1024,Available(i)) #for i <= 683,875,1333|
\end{Verbatim}
 

\subsection{\textcolor{Chapter }{InverseAvailableMap}}
\logpage{[ 2, 1, 2 ]}\nobreak
\hyperdef{L}{X798138987C8EBBB2}{}
{\noindent\textcolor{FuncColor}{$\triangleright$\enspace\texttt{InverseAvailableMap({\mdseries\slshape N})\index{InverseAvailableMap@\texttt{InverseAvailableMap}}
\label{InverseAvailableMap}
}\hfill{\scriptsize (function)}}\\
\textbf{\indent Returns:\ }
\texttt{int} 



 This function handles the translation between the available ordering of the
groups of order 1024 and the groups of order 1024. For $1 \leq i \leq 49,487,367,289$ if \texttt{SmallGroup(1024,i)} is available this will return its position in the available groups list or
else it will print a message telling you that it is not available and return
0. }

 
\begin{Verbatim}[commandchars=!@|,fontsize=\small,frame=single,label=Example]
  !gapprompt@gap>| !gapinput@InverseAvailableMap(AvailableMap(i)) = i;|
  !gapprompt@gap>| !gapinput@InverseAvailableMap(3568);|
  This is an immediate descendant of the elementary abelian group of order 32 and is not available
  0
\end{Verbatim}
 

\subsection{\textcolor{Chapter }{Heritage (for IsGroup)}}
\logpage{[ 2, 1, 3 ]}\nobreak
\hyperdef{L}{X78622CFD846746FD}{}
{\noindent\textcolor{FuncColor}{$\triangleright$\enspace\texttt{Heritage({\mdseries\slshape G})\index{Heritage@\texttt{Heritage}!for IsGroup}
\label{Heritage:for IsGroup}
}\hfill{\scriptsize (attribute)}}\\
\textbf{\indent Returns:\ }
\texttt{list} 



 Returns as a list the following information for a group of order 1024 loaded
from the library \texttt{[ParentGroupID, ParentGroupOrder, Step, Age]}. The age of a group is the position of the group among its siblings in the
ordered list of their standard PC codes. }

 }

 }

 \def\bibname{References\logpage{[ "Bib", 0, 0 ]}
\hyperdef{L}{X7A6F98FD85F02BFE}{}
}

\bibliographystyle{alpha}
\bibliography{manualbib.xml}

\addcontentsline{toc}{chapter}{References}

\def\indexname{Index\logpage{[ "Ind", 0, 0 ]}
\hyperdef{L}{X83A0356F839C696F}{}
}

\cleardoublepage
\phantomsection
\addcontentsline{toc}{chapter}{Index}


\printindex

\immediate\write\pagenrlog{["Ind", 0, 0], \arabic{page},}
\newpage
\immediate\write\pagenrlog{["End"], \arabic{page}];}
\immediate\closeout\pagenrlog
\end{document}
