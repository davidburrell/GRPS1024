% generated by GAPDoc2LaTeX from XML source (Frank Luebeck)
\documentclass[a4paper,11pt]{report}

            \usepackage{a4wide}
            \newcommand{\bbZ}{\mathbb{Z}}
        
\usepackage[top=37mm,bottom=37mm,left=27mm,right=27mm]{geometry}
\sloppy
\pagestyle{myheadings}
\usepackage{amssymb}
\usepackage[utf8]{inputenc}
\usepackage{makeidx}
\makeindex
\usepackage{color}
\definecolor{FireBrick}{rgb}{0.5812,0.0074,0.0083}
\definecolor{RoyalBlue}{rgb}{0.0236,0.0894,0.6179}
\definecolor{RoyalGreen}{rgb}{0.0236,0.6179,0.0894}
\definecolor{RoyalRed}{rgb}{0.6179,0.0236,0.0894}
\definecolor{LightBlue}{rgb}{0.8544,0.9511,1.0000}
\definecolor{Black}{rgb}{0.0,0.0,0.0}

\definecolor{linkColor}{rgb}{0.0,0.0,0.554}
\definecolor{citeColor}{rgb}{0.0,0.0,0.554}
\definecolor{fileColor}{rgb}{0.0,0.0,0.554}
\definecolor{urlColor}{rgb}{0.0,0.0,0.554}
\definecolor{promptColor}{rgb}{0.0,0.0,0.589}
\definecolor{brkpromptColor}{rgb}{0.589,0.0,0.0}
\definecolor{gapinputColor}{rgb}{0.589,0.0,0.0}
\definecolor{gapoutputColor}{rgb}{0.0,0.0,0.0}

%%  for a long time these were red and blue by default,
%%  now black, but keep variables to overwrite
\definecolor{FuncColor}{rgb}{0.0,0.0,0.0}
%% strange name because of pdflatex bug:
\definecolor{Chapter }{rgb}{0.0,0.0,0.0}
\definecolor{DarkOlive}{rgb}{0.1047,0.2412,0.0064}


\usepackage{fancyvrb}

\usepackage{mathptmx,helvet}
\usepackage[T1]{fontenc}
\usepackage{textcomp}


\usepackage[
            pdftex=true,
            bookmarks=true,        
            a4paper=true,
            pdftitle={Written with GAPDoc},
            pdfcreator={LaTeX with hyperref package / GAPDoc},
            colorlinks=true,
            backref=page,
            breaklinks=true,
            linkcolor=linkColor,
            citecolor=citeColor,
            filecolor=fileColor,
            urlcolor=urlColor,
            pdfpagemode={UseNone}, 
           ]{hyperref}

\newcommand{\maintitlesize}{\fontsize{50}{55}\selectfont}

% write page numbers to a .pnr log file for online help
\newwrite\pagenrlog
\immediate\openout\pagenrlog =\jobname.pnr
\immediate\write\pagenrlog{PAGENRS := [}
\newcommand{\logpage}[1]{\protect\write\pagenrlog{#1, \thepage,}}
%% were never documented, give conflicts with some additional packages

\newcommand{\GAP}{\textsf{GAP}}

%% nicer description environments, allows long labels
\usepackage{enumitem}
\setdescription{style=nextline}

%% depth of toc
\setcounter{tocdepth}{1}





%% command for ColorPrompt style examples
\newcommand{\gapprompt}[1]{\color{promptColor}{\bfseries #1}}
\newcommand{\gapbrkprompt}[1]{\color{brkpromptColor}{\bfseries #1}}
\newcommand{\gapinput}[1]{\color{gapinputColor}{#1}}


\begin{document}

\logpage{[ 0, 0, 0 ]}
\begin{titlepage}
\mbox{}\vfill

\begin{center}{\maintitlesize \textbf{ GRPS1024 \mbox{}}}\\
\vfill

\hypersetup{pdftitle= GRPS1024 }
\markright{\scriptsize \mbox{}\hfill  GRPS1024  \hfill\mbox{}}
{\Huge \textbf{ Utilities for iterating through the groups of order 1024 with p-class at least
3 \mbox{}}}\\
\vfill

{\Huge  0.0.1 \mbox{}}\\[1cm]
{ 20 June 2022 \mbox{}}\\[1cm]
\mbox{}\\[2cm]
{\Large \textbf{ David Burrell\\
   \mbox{}}}\\
\hypersetup{pdfauthor= David Burrell\\
   }
\end{center}\vfill

\mbox{}\\
{\mbox{}\\
\small \noindent \textbf{ David Burrell\\
   }  Email: \href{mailto://davidburrell@ufl.edu} {\texttt{davidburrell@ufl.edu}}\\
  Homepage: \href{https://davidburrell.github.io/} {\texttt{https://davidburrell.github.io/}}}\\
\end{titlepage}

\newpage\setcounter{page}{2}
\newpage

\def\contentsname{Contents\logpage{[ 0, 0, 1 ]}}

\tableofcontents
\newpage

     
\chapter{\textcolor{Chapter }{Groups of Order 1024}}\label{Chapter_Groups_of_Order_1024}
\logpage{[ 1, 0, 0 ]}
\hyperdef{L}{X850EDC61865B454E}{}
{
  
\section{\textcolor{Chapter }{Overview}}\label{Chapter_Groups_of_Order_1024_Section_Overview}
\logpage{[ 1, 1, 0 ]}
\hyperdef{L}{X8389AD927B74BA4A}{}
{
  

 This package gives access to all of the groups of order 1024 with p-class 3
and greater. The groups are sorted first by their parent group ids and then by
the pc codes of the standard presentations for the groups. These groups were
used in the 2021 enumeration of the groups of order 1024 \cite{Burrell2021a} and represent a complete list of the isomorphism classes of the groups of
order 1024 with p-class 3 and greater. 

 }

 }

   
\chapter{\textcolor{Chapter }{Functionality}}\label{Chapter_Functionality}
\logpage{[ 2, 0, 0 ]}
\hyperdef{L}{X87F1120883F5B4D0}{}
{
  

 
\section{\textcolor{Chapter }{Methods}}\label{Chapter_Functionality_Section_Methods}
\logpage{[ 2, 1, 0 ]}
\hyperdef{L}{X8606FDCE878850EF}{}
{
  

 This section will describe the functions available in GRPS1024 

 

\subsection{\textcolor{Chapter }{NumDescendants}}
\logpage{[ 2, 1, 1 ]}\nobreak
\label{CharacterDegreesOfBlocks}
\hyperdef{L}{X8434B096793033BC}{}
{\noindent\textcolor{FuncColor}{$\triangleright$\enspace\texttt{NumDescendants({\mdseries\slshape Order, ID})\index{NumDescendants@\texttt{NumDescendants}}
\label{NumDescendants}
}\hfill{\scriptsize (function)}}\\
\textbf{\indent Returns:\ }
an \texttt{int} 



 returns the number of immediate descendants of order 1024 of
SmallGroup(Order,ID) }

 

\subsection{\textcolor{Chapter }{LoadDescendants}}
\logpage{[ 2, 1, 2 ]}\nobreak
\hyperdef{L}{X7908BA5B84777B49}{}
{\noindent\textcolor{FuncColor}{$\triangleright$\enspace\texttt{LoadDescendants({\mdseries\slshape Order, ID})\index{LoadDescendants@\texttt{LoadDescendants}}
\label{LoadDescendants}
}\hfill{\scriptsize (function)}}\\


 Loads the immediate descendants of \texttt{SmallGroup(Order,ID)} into global variable \texttt{GRPS1024{\textunderscore}DESC.(Order)[ID]} if the descendants are available else it loads into global variable \texttt{GRPS1024{\textunderscore}ENUM.(Order)[ID]} the number of immediate descendants of order 1024 of \texttt{SmallGroup(Order,ID)} }

 

\subsection{\textcolor{Chapter }{CheckoutDescendants}}
\logpage{[ 2, 1, 3 ]}\nobreak
\hyperdef{L}{X82D11AA08717F710}{}
{\noindent\textcolor{FuncColor}{$\triangleright$\enspace\texttt{CheckoutDescendants({\mdseries\slshape Order, ID})\index{CheckoutDescendants@\texttt{CheckoutDescendants}}
\label{CheckoutDescendants}
}\hfill{\scriptsize (function)}}\\
\textbf{\indent Returns:\ }
'list' 



 Returns the immediate descendants of \texttt{SmallGroup(Order,ID)} as a list. If the list is empty this implies that the immediate descendants of \texttt{SmallGroup(Order,ID)} are not available, this might be because it doesn't have any or that \texttt{SmallGroup(Order,ID)} has $p$-class 1 and the presentations are not available. To see if the group has
immediate descendants use \texttt{NumDescendants(Order,ID)}. }

 

\subsection{\textcolor{Chapter }{IsAvailable}}
\logpage{[ 2, 1, 4 ]}\nobreak
\hyperdef{L}{X7C8C9C437C9A6BED}{}
{\noindent\textcolor{FuncColor}{$\triangleright$\enspace\texttt{IsAvailable({\mdseries\slshape N})\index{IsAvailable@\texttt{IsAvailable}}
\label{IsAvailable}
}\hfill{\scriptsize (function)}}\\
\textbf{\indent Returns:\ }
\texttt{true} or \texttt{false} 



 Checks if the Nth group of order 1024 is available, there are 49487367289
groups of order 1024 and those which have $p$-class three and greater are available through this package. }

 

\subsection{\textcolor{Chapter }{FindGroupN}}
\logpage{[ 2, 1, 5 ]}\nobreak
\hyperdef{L}{X7A118404838C8EC7}{}
{\noindent\textcolor{FuncColor}{$\triangleright$\enspace\texttt{FindGroupN({\mdseries\slshape N})\index{FindGroupN@\texttt{FindGroupN}}
\label{FindGroupN}
}\hfill{\scriptsize (function)}}\\
\textbf{\indent Returns:\ }
list if available 



 Finds the $N$th group of order 1024 in storage and returns the group. If the group is not
available then an informative message about the groups heritage is printed. }

 

\subsection{\textcolor{Chapter }{FindNthAvailableGroup}}
\logpage{[ 2, 1, 6 ]}\nobreak
\hyperdef{L}{X87089300865B4A5E}{}
{\noindent\textcolor{FuncColor}{$\triangleright$\enspace\texttt{FindNthAvailableGroup({\mdseries\slshape N})\index{FindNthAvailableGroup@\texttt{FindNthAvailableGroup}}
\label{FindNthAvailableGroup}
}\hfill{\scriptsize (function)}}\\
\textbf{\indent Returns:\ }
'FindGroupN(AvailableMap(N))' 



 Of the available groups returns the $N$th one. This should be the main function used to iterate throught the $683,875,133$ available groups. }

 

\subsection{\textcolor{Chapter }{AvailableMap}}
\logpage{[ 2, 1, 7 ]}\nobreak
\hyperdef{L}{X7E989FA979E72135}{}
{\noindent\textcolor{FuncColor}{$\triangleright$\enspace\texttt{AvailableMap({\mdseries\slshape N})\index{AvailableMap@\texttt{AvailableMap}}
\label{AvailableMap}
}\hfill{\scriptsize (function)}}\\
\textbf{\indent Returns:\ }
'int' 



 This takes handles the translation between the ordering of the groups of order
1024 and the available groups of order 1024. For $1 \leq i \leq 683,875,133$ this will return the position of the $i$th available group among all the groups of order 1024. }

 }

 }

 \def\bibname{References\logpage{[ "Bib", 0, 0 ]}
\hyperdef{L}{X7A6F98FD85F02BFE}{}
}

\bibliographystyle{alpha}
\bibliography{manualbib.xml}

\addcontentsline{toc}{chapter}{References}

\def\indexname{Index\logpage{[ "Ind", 0, 0 ]}
\hyperdef{L}{X83A0356F839C696F}{}
}

\cleardoublepage
\phantomsection
\addcontentsline{toc}{chapter}{Index}


\printindex

\immediate\write\pagenrlog{["Ind", 0, 0], \arabic{page},}
\newpage
\immediate\write\pagenrlog{["End"], \arabic{page}];}
\immediate\closeout\pagenrlog
\end{document}
